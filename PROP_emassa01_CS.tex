\documentclass[10pt]{article}
\setlength{\topmargin}{-0.8in}
\setlength{\textheight}{9.5in}
\setlength{\oddsidemargin}{-.13in}
\setlength{\textwidth}{6.5in}

\usepackage{multirow}
\usepackage{array}
\newcolumntype{L}{>{\centering\arraybackslash}m{3cm}}

\usepackage{graphicx}
\graphicspath{{images/}}

\begin{document}

\title{User Modeling and Personalisation in Search for People with Autism}
\author{Esha Massand\\
MSc Computer Science Project Proposal\footnote{This proposal is substantially the result of my own work, expressed in my own words, except where explicitly indicated in the text. I give my permission for it to be submitted to the JISC Plagiarism Detection Service. }}
\date{\today}
\maketitle

\begin{abstract}
The current proposal presents my objective to research and build a user model and web application for individuals on the Autism Spectrum within Search. The model will be built around the core features of Autism. The model will be applied to results returned from a synthesis of three leading existing search engines. The web application will be integrated with motion-controlled user interfaces (UI). The project will provide novel insights into the needs and wants of individuals with Autism within search, and enable future development and interventions within these information streams and communication channels.
\end{abstract}

\tableofcontents

\section{Introduction \& Background}
\subsection{Problem Statement} \label{prob}
Most search engines apply user models to refine user search queries. These models can often lack specificity for users/groups of users \cite{usermodel}. Although adaptive search engines have now become prominent; they use search history, locale, and demographics to match results to the user's intentions, these models make several a priori assumptions about users from specific subgroups. No attempts have been made to define a user model within search for individuals with Autism, so it is currently unknown whether current user models need to be adjusted for this subgroup. We argue that general user models, and the needs of the user with Autism differ, and the project aims to build a user model of Autism to address this issue.\\
As it is well established that individuals with Autism are more engaged when using technology that is receptive and interactive (e.g., games, responsive consoles, motion controlled devices) compared to technology that is not \cite{motioncontrollerforautism}, this project will combine interactive, motion recognition hardware with Search to improve the UI (user interface) and architecture of Search for individuals with Autism.

\subsection{What is the Autism Spectrum?}
Autism is amongst the most common neurodevelopmental condition and it is currently estimated that 1/68 children meet criteria for Autism Spectrum (CDC, 2014). Autism is five times more common amongst boys than girls (1/42 boys, and 1/189 girls). According to the DSM-V (2013) diagnostic manual, Autism is characterized by persistent and early deficits in reciprocal social interaction and repetitive behaviours. Individuals vary from high functioning to low functioning (along a spectrum), with behaviours emerging around 2 to 3 years of age. 

\begin{center}
\includegraphics[scale=0.5]{asd}\\
The Autism Triad \cite{triad}
\end{center}

\subsection{The Role of Context In Search} \label{the problem}
It is unlikely that any given page on the web will contain a word or phrase that means exactly (or nearly) the same as another word or phrase in that language (e.g., shut and close). How is it then that your search engine picks these phrases to mean the same thing, and returns them synonymously in the results of your query? Well, quite simply put, it is by virtue of the fact that each of their neighbouring words and associations are similar. These are indirect, higher-order associations, and provide the context in which the search engine can index keywords. This context plays a crucial role in search; first, in the interpretation of the user query, and second, it is reflected in the results returned to the user.\\
Search-engine algorithms assume that the user is context-driven, and attempt to model the user's intent using higher-order contextual information gathered from available web pages. This process also models the brain's ability to extract context and semantic associations from information.\\

 People with Autism are less context-sensitive and prefer a more detail-focused processing style \cite{mottron}, and form search queries very differently. Individuals with Autism are also less likely to engage in a relational (hierarchically organized) style of processing \cite{bowler} suggesting that relating information in a hierarchically organised framework is less likely. Hierarchical organisation implies a great deal of flexibility and mental-shifting, as a simple example, in a search for 'apple', it would imply awareness that the word is related to 'pear' but also to 'fruit'. Awareness of this latter relation also suggests awareness that 'apple' is related to 'pomegranate'. This is of course, a simple example, but these associations can get very complex very quickly. Generally speaking, individuals with Autism prefer, and are more likely to engage in an item-specific processing style, and, whilst intelligent cognition is definitely possible, search queries are more likely formed of first-order associations\footnote{Of course there is a great deal of individual variability in the Autism Spectrum.}. \\

\subsection{What should Search offer people with Autism?}\label{What should Search offer people with Autism}
\subsubsection{Search and Learning}
The Internet is one of the largest resources of information. Search engines allow users to collate hundreds of links on a single topic, using only a few words or phrases. The user sorts the returned results into 'relevant' or 'irrelevant' categories, flexibly shifting (mentally) between one result and the next, to determine the relevance of each page returned by the search engine\footnote{In the typical case}. Search allows the user to assimilate the information on the page into their knowledge and is an important learning tool; its significance is duly noted because of the learning benefits it brings for children and adolescents as they begin to navigate the Internet and gain an understanding of several subjects. 

\subsubsection{Clues from virtual reality and gaming}
Almost all teenagers (97\% of those aged 12-17) use a computer, web, portable or console device, 73\% of which is desktop/laptop based. Teenagers with Autism also use technology and spend a substantial amount of their time using devices \cite{Shane and Albert}. For individuals with Autism, computer-based technologies can provide a stable, consistent learning environment that can be customized \cite{moore}. 

\subsubsection{Motion Controllers}
Motion recognition devices can be programmed to make consistent responses to environmental triggers. This is unlike real-world situations where environmental responses are not always consistent and may require further interpretation or ‘guess-work’. These controlled and interactive environments have shown promise for improving social communication skills and reducing repetitive behaviours \cite{gameshealth}.

\subsubsection{Visual not text-based}
People with Autism demonstrate stronger visual memory \cite{fabienne} than verbal memory, so a more visually-oriented approach to search (reducing the 'working memory' load \cite{workingmem}) is a more appropriate way to present data to bolster the strength of visual memory in people with Autism.

\subsection{Existing Combination/Advanced Search Engines}\label{Existing Combination/Advanced Search Engines}
The first part of this project involves the synthesis of results from three leading search engines. Current existing solutions are: 

\subsubsection{Bing vs Google}
'Bing vs. Google' presents the users’ search query results from both search engines, allowing the user to make a comparison, and provides the experience of navigating both search pages simultaneously.  
The number of other personal preferences options is very limited, and there is double the information on the page (verbal overload).

\subsubsection{Qrobe.it}
Qrobe combines three search engines’ results (Google, Bing and Ask) and presents them conveniently on one page. Unlike Bing vs Google the user can search ‘web’, ‘images’ or ‘popular’. Qrobe has not got an open (or well tested) API for developers to use to extend its functionality further, meaning it is a riskier option to choose for this project.

\subsubsection{AskBoth}
AskBoth – is a work in progress, and combines both Google and Bing, with a section in the middle dedicated to twitter. AskBoth argues that the selling points for the site are it’s ‘uncomplicatedness’, aesthetics and user experience (UX) – which promises to be particularly good (promised, since 2009!).

\subsubsection{Spectra}
Spectra (results from Google, Bing and Yahoo), allows users to assign weights and determine the way results are displayed. Spectra gathers the search results, ranks them and displays them according to their algorithm. Spectra does not provide an API for developers, and the rigor of the search is hard to test, (not much user data available to analyse).

\subsubsection{Conclusions and ways forward}
These search engines allow users to see more results than what one search engine alone would present. Bing vs Google, and AskBoth, do this at a cost -- redundancy (near-duplicates) and ‘cognitive overload’. This is not ideal for users with Autism, as this is precisely the opposite of what the application's aims and objectives were (see Section~\ref{prob}). Spectra and Qrobe, do not offer an Open API and are not suitable solutions for the current project. \\
I will work on the creation and synthesis of the results using the Google, Bing and Yahoo API's, and incorporate a user model of Autism.

\section{Aims and Objectives} 
It has been suggested in the Psychology literature that individuals with Autism best assimilate new information when it is: concrete (not abstract); presented in contextually-relevant chunks; not verbally "overloaded" (not too many words); and presented in a set visual format. Given that several interventions have been built around this body of literature (see www.autismspeaks.org), we can make adjustments to current Search to enhance its benefit for people with Autism. For example, results can: be assessed for the Key Word In (a suitable) Context; use a similar order of semantic association, in line with the search query itself (precedence for first-order relations); have smaller snippets and be presented in a more manageable way (less overloaded with words); have visual consistency; have a high degree of verbal consistency/similarity with the search query.
\\As a whole, the user model will refine search results based on the presence of first-order, or, item-specific relations to a search query (rather than hierarchical relations). The model will be developed around well-understood cognitive processes in Autism. Other elements of the user model will be decided with user-feedback (during testing). 
\\This project will integrate these insights from the Psychology literature with the proposed application.

\subsection{Proposed Search and Controller Web Application}\label{proposed}
Below I list the core and non-core features of the Search tool:

\begin{center}
\includegraphics[scale=0.7]{searchEngArchi}
Proposed application: The User Model will be applied to existing Search Engine Architecture\cite{seimage}, and be integrated with a Motion Controlled interface.
\end{center}


\subsubsection{Core Features}
\begin{enumerate}
\item  A web application that synthesises the results from three of the largest and most popular search engines (Google, 67.5\%, Microsoft Bing 18.4\% and Yahoo 10.3\%) \cite{adam}

\item The design and implementation of a stereotyped user model of Autism to filter user search results

\item Key Word In Context search; returning verbally-concrete, not text-overloaded results, in a consistent manner. 

\item Prioritisation of results which have first-order semantic relations to the query words (see Section~\ref{the problem}), i.e., they appear in matched context to the search query.

\item Motion controlled UI (see Section \ref{hardware}).
\end{enumerate}

\subsubsection{Non-core Features}
\begin{enumerate}

\item Implement a higher ranking for pages with most images. 
\item Use of WebGL and the three.Js library for 3D UI .
\item Compatibility with other motion controller devices (e.g., head-mounted devices).

\end{enumerate}

\section{Plan for Developing the Solution}

\subsection {Creating a User Model of Autism}\label{usermodel}
\subsubsection{What is a user model?}
User modeling has strong implications for human-computer interaction (HCI); by creating a representation of the user, the system can be better informed about how to behave in various circumstances, e.g., the user’s demographics, needs, preferences, likes, dislikes, goals, plans, knowledge, and skill. A user model is a collection of information associated with a particular user (usually a data structure), with which a system can adapt/customise its behaviour in line with the user’s needs. The information can be gathered when the user 'sign's up' for the service (e.g., Google+, Facebook), and starts the formation of a data model. \\

\subsubsection{Types of user models}
User models can be static, and unchanging (i.e., no algorithms are used in order to teach the model about the changing preferences of the user, and no new information is fed into the model), or dynamic (representation of the user with their up-to-date changes in interests, and recent interactions with the system). User models may also be stereotyped. This is when the system infers or assumes characteristics about a user from data gathered from other users within that distinct subset. Lastly a user model can be highly adaptive and try to model the one user on their own, without stereotyping or inferring the characteristics of the user, however this requires a large amount of data collection before implementation.\\
A stereotyped user model will be used. Basic information will be gathered via a registration process, and added to a user model of ASD. The benefits are that the model can be built quickly using clusters of characteristics of individuals with Autism.

\subsubsection{Adaptive / Personalised Search}
Adaptive or Personalised search associated each user with a HTTP-cookie containing information such as login information (gender, age), preferences (languages, interests) and other information about previous searches based on site traffic. This allows the website to ‘remember’ what buttons the user clicked on, or what sites they visited. This cookie record allows the search engine to return results that are highly relevant to the search query, but also highly relevant to the pages that the user visited through previous searches. 

\subsubsection{Disadvantages of Current Adaptive Search for Individuals with Autism}
Although adaptive search seems to have significant user benefit in terms of relevance to the user for that search query, it decreases the likelihood that the user encounters new information and biases the results towards the users location and their previous site traffic.  This has the unwanted effect of creating a filter bubble (Pariser, 2011), which is argued to close us off from important and relevant information and create a personal ecosystem of information for one particular user, creating the impression that “our narrow self interest is all that exists”. The filter bubble also has potential privacy problems, as the user may be unaware that the search has been specifically tailored towards their interests and they wonder why things that they have previously searched for have become more and more relevant. There are search engines that have attempted to address this unwanted effect, by not tracking or saving user information (e.g., DuckDuckGo.com). As users are not linked to their search queries, it limits them being targeted by adverts related to their previous searches. Unfortunately the filter bubble may positively reinforce restricted interests in Autism as the user constantly receives feedback about their previous (idiosyncratic and personalized) searches without being able to break out of that repetitive loop. \\Recent research has suggested personalization also increases ‘background noise’ relative to the search results \cite{briggs}. Briggs (2014) suggests that there is a carry-over effect in personalized search for the users, whereby prior search results influence the results of subsequent searches. It should be noted that personalization of search results generally takes a lower priority for the ranking algorithms than the URLs ranked top in terms of their relevance for the search query. Nevertheless this carry-over may be particularly disadvantageous for people with Autism (some of whom already have restricted and repetitive interests) as it muddies their search space.\\
In order to produce a search tool specifically tailored to reduce the filter bubble effect in Autism, widen the information gateway and reduce the possibility for restricted and repetitive searches, the weighting on previous search results needs to be reduced. This is something I will investigate in the project, particularly for individuals with restricted interests. For these users, it would limit the possibility that they get trapped in a spiraling loop of ever-narrowing user-relevant information and over personalization of self-reinforced information ecosystems.

\subsubsection{Persisting the User's Information}
The user will be asked to sign in with a Google+ account, and their data will be linked to the web application. The Google+ API includes methods to access 4 resource  types; People, their Activities, Comments and Moments. A person is represented with many fields in Google +, including name, gender, title, occupation, all of which can be used to model individual users in the current project. Information about web searching history for any individual user can also be obtained from the browser history. 

\subsection{Implementing the Application}\label{api}
\subsubsection{Selecting Search Engines} 
The three most popular search engines (as calculated using an average of the unique monthly visitors) are Google (1,100,000,000), Bing (350.000.000) and Yahoo! (300,000,000)\cite{ebiz}. Google is the goliath question-answering system (query volume = 64.5\%)\cite{adam}, and is often considered the most innovative and dynamic. It is popular amongst users worldwide (using global traffic rank figures, in March 2015). Yahoo (2003) was the first ever web directory service; it has stronger advertising and e-commerce partnerships and has a query volume of 19.8\%. Bing was officially launched in 2005, and has a query volume of 12.8\%, which is substantially less than Google, but nevertheless, is within the top 3 search engines. Other search engines were not included, to limit redundancy of the search results returned (see Section~\ref{Existing Combination/Advanced Search Engines}). 

\subsubsection{APIs, Text-Search Libraries}
I will use API’s provided by Google, Yahoo and Bing, as they are far more efficient than inspecting the source code for the search results page (e.g., Apache Lucene Key Word In Context is optimised for maximum search efficiency see Section~\ref{apache}). 

\subsubsection{Google Custom Search}
Google Custom Search (GCS) provides a Java API to create a personalised search engine that can be configured to search web pages and images. It works on a pay per search principle. Once signed up, the GCS requires a consumer key and secret, which are hardcoded in the development of the search.\\
The API has methods which (amongst other things) allows the extraction of image search results, page dates, formatting dates, custom snippets, sort by and filter methods. However, this may not be enough, and the GCS API may need to be used in conjunction with a textual-search library in order to reach the goals of this project (the API does not offer Key Word In Context Search in order to trace the contextual information being returned to the user). Costs \$0.01/search.

\subsubsection{Yahoo BOSS}
Just like Google Custom Search, the Yahoo BOSS Java API required the creation of a search engine project (pay per search) with a consumer key and secret. The API is also easy to use and offers the same functionality as the CSC but again is not sufficient alone to reach the goals of the project. Costs \$0.01/search.

\subsubsection{Bing Search API (Data)}
The Bing Search API, similar to Yahoo BOSS and GCS will produce results for Web, Images, News, Videos, Related Search. Bing Search Java API also includes spelling suggestions based on the query entered. Costs \$0.00/search (max 5000 searches/month)

\subsubsection{Faroo API}
Is a free alternative Java API to Google Custom Search API (business), Yahoo BOSS API  (commercial) and Bing Web Search Enterprise (commercial). It offers the possibility to do a Web Search with more that 2 billion pages indexed. Faroo can return news search (articles from newspapers, magazines and blogs) and sort results by publishing date, with author and article image. Trending news pages are also indexed and can be grouped by topic. The API includes suggestions with auto-completes for misspelled items in the search query \cite{faroo}.

\subsubsection{Apache Lucene Library}\label{apache}
Apache Lucene Library is a text-based context search. It is particularly relevant for the current project because it provides powerful, accurate and efficient algorithms to search textual data, the algorithms are scalable and high performance, so will enable users to receive results from their search query with good speed. The API offers the possibility to carry out phrase, wildcard, proximity and range queries which will mean the goals of the project can be fulfilled (Package org.apache.lucene.search.highlight, for the aims of the project refer to ~Section \ref{the problem}). The library also affords ranked searches, with type tolerant suggesters and field searching.

\subsubsection{Key Word In Context} \label{KWIC} 
The Apache Lucene open-source search engine library written in Java allows contextual-text search, also known as Key Word In Context (KWIC)\cite{kwic}. KWIC works by forming an index to allow each word to be searchable. The library takes care of the efficiency of this process, and can return weighted terms of a given query (as an example).


\subsubsection{Google+ API}
Persona (a type of user) development will support the user modeling process by identifying particular characteristics of individuals with Autism in Search. An individual’s personal information pertaining to the persona, will be stored in a Google+ user profile and can be used with the Google+ API (written in Java), containing information such as age, gender, lifestyle, frequent tasks, tools used, and the resources they commonly use. It is also be possible to store information in the 'about me' section on the profile about individual diagnosis (Autism, Asperger, and high/low functioning). This information can be parsed when the query is submitted to the search engine.

\subsubsection{API for 'LEAP' Motion Controller}
LEAP Motion SDK offers an API to get tracking data from the LEAP Motion Service. A WebSocket interface, allows LEAP Web Based applications, and a WebSocket server listening in on http://127.0.0.1:6437. The user can enable or disable the WebSocket server as they choose to do so, in the device's control panel.
The server sends tracking data in JSON messages and an application can send configuration messages back. This library will be used to establish connection to the server and consume the JSON messages \cite{leap}. 


\subsubsection{ThreeJs Library (Non-Core)}
This javascript library enables WebGL-3D in a web browser. WebGL brings hardware-accelerated 3D graphics to the browser without installing additional software. This library may be used to better integrate the application with the motion controller, and improve the experience of embodiment, and UI of the application.


\subsection{Integrating the Application with Motion Controller Hardware }\label{hardware}
\subsubsection{Hardware Selection Process}
The usability of the hardware will be determined as follows: 
\begin{enumerate}
\item Good timing of the device correlates to a good meaning and a good UX. The LEAP has options to ‘poll’ frames at a constant rate (to keep timing of movement accurate).
\item Cognitive ‘lag’ time. Each of our senses operates with a different lag time. Hearing has the fastest sense-to-cognition/understanding time; sight is the slowest. The device should therefore work with the combinatorial configuration of the senses.
\item As this is a tool to be used with individuals with Autism, the sensory experience of the device cannot be overwhelming.
\item Cognitive-load should not be high (the device is being used to assist with search, so operating the device should not require a great deal of cognitive effort).
\item The device should integrate with concrete behaviours, e.g., drop or grab. 

\end{enumerate}

\subsubsection{LEAP Controller}
The leap controller can recognize and track hands, fingers and finger-like tools. It can report positions, motions and gestures using an infrared light and optical sensors along the x, y and z axes (Cartesian coordinate system). The controller has a 150-degree field of view, and can operate in a range of 1 inch to 2 feet. The API works with distance in millimetre resolution. Time is measured in microseconds, speed in mm/s and angles in radians.

\begin{center}
\includegraphics[scale=0.5]{leap}\\
The LEAP controller, with 150 degree view \cite{leap}.
\end{center}

The LEAP uses frames to represent tracked entities such as hands, fingers, tools or gestures. Motion data is recorded as a set of frames (stored, read-only) containing the detected information. 
Frames can be created by calling the Controller.frame(), and up to 60 can be held in the history buffer with the current API. Frames may be 'dropped' if there are resource contrainsts, or, they are missed for example. Once a frame is created, the data can be gathered from the hands(), arms(), fingers(), tools() methods.

\subsubsection{Hands}
The Hand class, returns information about the ID, position of fingers associated with that hand, and arm infomration (left/right).

\begin{center}
\includegraphics[scale=0.6]{palm}\\
The Hand palmNormal() method and direction vectors define the orientation of the hand \cite{leap}\\.
\end{center}

The software uses parts of visible hand, internal model and previous observations to form a model of the hand. Five finger positions will always be shown but subtle movements of hand, especially if they are tucked up into the hand are harder to detect. For this there is a Hand.confidence() method that provides a rating of how well the observed data fit the internal model \cite{leap}.

\subsubsection{Arms}
The Arms class can return information about orientation, length, width and end points of movements. The LEAP controller software bases these return measurements on previous observations of the user, and using typical human proportions.

\subsubsection{Fingers}
These characteristics are based on the anatomy of the hand, and recent observations. 

\begin{center}
\includegraphics[scale=0.7]{fingers}\\
Finger tip position and direction are given as vectors. \cite{leap}.
\end{center}


\subsubsection{Tools \& Pointables}
Tools can represent any real object (noun), but are longer, straighter and thinner than fingers. Tools must be cylindrical.

\subsubsection{Gestures}
The LEAP recognises certain movement patterns (for each finger or tool individually) allowing the user to indicate an intent. These gestures are observed in a frame and include: CircleGesture, KeyTapGesture, ScreenTapGesture and SwipeGestures.


\subsection{Project Plan}\label{plan}

\subsubsection{Project Timeline}
A high level plan of the project timeline is presented in table ~Table \ref{stages}. The start date of the project is June 5th 2015, and end date is September 13th 2015.
\begin{table}[h]
\caption{Project Stages} 
\centering
\begin{tabular}{ L |c| L}
\hline\hline 
Dates & Task & Priority\\ [0.5ex]
\hline 
Jun 05 - Jun 12 & Gather relevant API's \& Libraries & MUST\\
\hline 
Jun 13 - Jun 19 & Work on synthesis of search results & MUST\\
\hline 
Jun 20 - Jun 27 & Research \& build user (and data) model of Autism & MUST\\
\hline 
Jun 28 - Jul 06 & Work on configuration with Google+ API & MUST\\
\hline 
Jul 07 - Jul 14 & Apply user model to Search& MUST\\ 
\hline 
Jul 14 - Jul 21 & Develop UI & MUST\\
\hline 
Jul 21 - Jul 28 & Integrate motion controller tools& MUST\\
\hline 
Jul 29 - Aug 05 & Develop questionnaire and eye-tracker set up & MUST\\ 
\hline 
Aug 06 - Aug 13 & Test the model and ask for user feedback & MUST\\
\hline 
Aug 14 - Aug 21 & Revise the user model and UI & MUST\\
\hline 
Aug 14 - Aug 21 & Develop UI with other motion controllers & COULD\\
\hline 
Aug 22 - Aug 29 & Develop UI using WebGL/threeJs library & COULD\\
\hline 
Aug 22 - Aug 29 & Write up project report & MUST\\ 
\hline 
Early Sep (tbc) & Present findings to supervisor & MUST\\
\hline 
Sep 13 & Submit report & MUST\\[1ex]
\hline
\end{tabular}
\label{stages} 
\end{table}

\subsubsection{Methodology}
The current project has a relatively short deadline in which a single developer will research and deliver a system prototype and report. The APIs, technology and areas of development are unfamiliar. As the final product depends on user feedback testing, there is an element of uncertainty about what the final product will be/should look like, i.e., a feature could be added/removed at the feedback stage. The characteristics of the current project mean that the most suitable methodology to deliver the application is Agile Methodology. I will focus on development and rapid feedback early in the development, to make changes to the project direction. This methodology works well with the demands and offers the most flexibility and adaptability.


\subsubsection{Development Languages}
I will use Eclipse text editor and attempt to make the application compatible with Google Chrome browser (as it is WebGL-compatible and may be useful for 3D interfaces). The Google, Yahoo, Bing and Apache Lucene APIs are available in Java. The LEAP SDK sends Frame information in JSON format to Web Browsers. As a non-core feature, I may use a 3D interface with the three.Js library which is also JSON format. I will use HTML5 for the development of the Web Application itself. I will use Git for Version Control, JUnit, JSON Test (for testing code) and Mockito (for testing when there are user/external dependencies). \\
JSON is light weight, language-independent data format, and a good tool for sharing data. Importantly JSON offers faster execution and server-side parsing by storing the data in arrays, so that the transfer of data is faster. Faster parsing is particularly important for sharing the LEAP motion controller data. Some of the drawbacks of JSON are that it only has limited support tools available, and little error handling capabilities. It is also vulnerable as it returns responses in wrapped function calls which are vulnerable to attack. Java is a platform and operating-system independent language. It offers a simple, dynamic and robust object oriented, functional language.

\subsubsection{Testing}
\subsubsection{User Testing}
To ascertain whether the goals of the project have been met I will need to test the application with people with Autism. This will include;
\begin{itemize}
\item Recruiting participants to take part in the research. Adolescents and adults with a diagnosis of Autism Spectrum. Recruited from NAS, will be asked to test the application.
\item Obtaining user feedback (relevance/explicit feedback, and, implicit feedback, e.g., mouse-clicks) on the initial product by testing the web search with the LEAP motion controller, with a group of individuals diagnosed with Autism. I will design a questionnaire to test the application's feasibility. If time allows, I hope to use a Tobii Eye-tracker TX300 (in the Department of Psychology, Centre for Brain and Cognitive Development, Birkbeck University of London) to gather high resolution eye-tracking data on the participants as they use the application. This will inform my future developments for revising the application.
\item Revising the model and the ideas to choose the best possible approach/tools to achieve optimization for people with Autism.
\end{itemize}

\subsubsection{Unit Testing}
For testing the application code, I will be using Test Driven Development (TDD). For Java I will use the current version of JUnit (at the time of writing this is 4.12).External dependencies will be mock tested using Mockito. JSON Test will be used to test JavaScript Object Notation \cite{jsontest}. As well as unit testing, Regression testing will be used to testing the project as a whole unit.

\subsubsection{Risks/issues, probabilities and mitigation of impact}
The possible risks associated with the projects development, their impact and how I will attempt to mitigate these risks is outlined in ~Table \ref{risks}. 
\begin{table}[h]
\caption{Risks \& Impact Mitigation} 
\centering
\begin{tabular}{|c | c | L | L |}
\hline\hline 
Liklihood & Impact & Risk & Mitigation\\ [0.5ex]
\hline 
LOW & HIGH & API's require significantly high payment & Find/use alternative\\
\hline 
LOW & HIGH & KWIC library does not offer methods needed to achieve goals of contextual text search & See if I can implement the method, or find additional API\\
\hline 
LOW & MEDIUM & Cannot get research participants with Autism to take part in a usability test & Expand age range of interest in attempt to find participants\\ 
\hline 
MEDIUM & MEDIUM & Google+ API does not configure a user persona/user model well & Use Facebook or alternative\\
\hline 
MEDIUM & HIGH & LEAP does not integrate with web application & Investigate user forums, contact LEAP to source answers, adjust web application accordingly.\\
\hline
MEDIUM & HIGH & User Model lacks power and users are unhappy with the Search system & Gather feedback for a further iteration. Modify weights of parameters in the algorithm. Revise the model. \\
\hline
MEDIUM & HIGH & Not enough turn around time to implement the feedback & Develop plan and prototype in time for report submission\\[1ex]
\hline
\end{tabular}
\label{risks} 
\end{table}


\subsection{Summary \& Concluding Statement}\label{future}
I have proposed to research and build a User Model within Search for people with Autism. I describe a model (based on well-understood aspects of cognition in Autism) to apply to search results, using text and content based libraries which will refine search results for these individuals. I hope these insights will assist with the forthcoming information-overload problem by exploiting these user models to turn the masses of information available into a specific set of “information goods”. 

\section{Appendices}
\subsection{Abbreviations}
\begin{tabular}{l l }
API & Application Programming Interface\\
ASD & Autism Spectrum Disorder\\
DSM & Diagnostic and Statistical Manual\\
HCI & Human Computer Interaction\\
UI & User Interface\\
UX & User Experience\\
\end{tabular}

\begin{thebibliography}{100}

\bibitem {gameshealth} Games for Health (2012) \textit{Screen-based technologies and Autism. 1}: 248-53

\bibitem {usermodel}Shen, X., Tan, B. and Zhai, C. (2005) Implicit User Modeling for Personalized Search, \textit{Conference on Information and Knowledge Management}, Bremen, Germany.

\bibitem{motioncontrollerforautism} Garzotto, F., Valoriani, M. and Bartoli, L. (2014), Touchless Motion-Based Interaction for Therapy of Autistic Children, Virtual, Augmented Reality and Serious Games for Healthcare, \textit{Intelligent Systems Reference Library, 68}, 2014, pp 471-494

\bibitem{moore}Moore, D. J., McGrath, P., \& Thorpe, J. (2000). Computer aided learning for people with autism—a framework for research and development. \textit{Innovations in Education and Training International, 37}, 218–228.

\bibitem{workingmem}Baddeley, A.D., \& Hitch, G. (1974). Working memory. In G.H. Bower (Ed.), \textit{The psychology of learning and motivation: Advances in research and theory, 8}, 47–89. New York: Academic Press.

\bibitem{leap} Leap Motion, \textit{Java SDK Documentation}, https://developer.leapmotion.com/documentation/java/index.html Retrieved 1 April 2015.

\bibitem{briggs}Briggs, J. \textit{A Better Understanding of Personalized Search}. https://www.briggsby.com/better-understanding-personalized-search/, Retrieved 5 April 2015.

\bibitem {Brusilovsky} Brusilovsky, P. and Tasso, C. (2004) User modeling for Web information retrieval. \textit{User Modeling and User Adapted Interaction, 14}, 2-3, 147-157.


\bibitem {CDC}Developmental Disabilities Monitoring Network Surveillance (2010) \textit{Centers for Disease Control and Prevention (CDC). Prevalence of autism spectrum disorders: Autism and Developmental Disabilities Monitoring Network, United States. MMWR Surveill Summ.2009; 58}, 10:1–20


\bibitem {Shane and Albert}Shane, H. C. and Albert, P. D. (2008) Electronic screen media for persons with autism spectrum disorders: results of a survey. \textit{Journal of Autism Developmental Disorders, 38},8 :1499-508. doi: 10.1007/s10803-007-0527-5.


\bibitem{jsontest}jsontest \textit{JSON Test}, http://www.jsontest.com/, Retrieved 8 April 2015.

\bibitem{triad}National Autistic Society, \textit{The Autistic Spectrum}, http://media.kingdown.wilts.sch.uk/mod/page/view.php?id=7374, Retrieved 20 March 2015

\bibitem {Lasater}Lasater, M. W., \& Brady, M. P. (1995). Effects of video self-modeling and feedback on task fluency: A home-based intervention. \textit{Education and treatment of children, 18}, 389-407.

\bibitem {Pariser} Pariser, E (2011) First Monday: What's on tap this month on TV and in movies and books, \textit{The Filter Bubble by Eli Pariser}. USA Today. Retrieved April 7, 2015. 

\bibitem{fabienne}Fabienne Samson, Laurent Mottron, Isabelle Soulières, Thomas A. Zeffiro (2011). Enhanced visual functioning in autism: An ALE meta-analysis. \textit{Human Brain Mapping} DOI: 10.1002/hbm.21307

\bibitem{kwic}Manning, C. D., Schütze, H. (1999) \textit{Foundations of Statistical Natural Language Processing}, p.35. The MIT Press.

\bibitem{seimage} Pray, S. \textit{Inverted Index : The basic ingredient behind the recipe called Search Engine}https://insightsdelight.wordpress.com/2012/01/24/inverted-index-the-basic-ingredient-behind-the-recipe-called-search-engine/, Retrieved 8 April 2015

\bibitem {Economist} Pariser, E. (2011) \textit{Invisible sieve: Hidden, specially for you}. The Economist. Retrieved April 8, 2015. Mr 

\bibitem {MacDuff} MacDuff, Krantz, \& McClannahan (2001). Prompts and prompt-fading strategies for people with autism. In C. Maurice, \& G. Green (Eds.), \textit{Making a difference: Behavioral intervention for autism}, 37-50. Austin, TX: Pro-Ed.


\bibitem {Sherer}Sherer, M., Pierce, K. L., Paredes, S., Kisacky, K. L., Ingersoll, B., Schriebman, L. (2001). Enhancing conversation skills in children with autism via video technology. \textit{Behavior Modification, 25}, 140-158.


\bibitem {Thiemann}Thiemann, K. S., \& Goldstein, H. (2001). Social stories, written text cues, and video feedback: Effects on social communication of children with autism. \textit{Journal of Applied Behavior Analysis, 34}, 425-446.

\bibitem{ebiz}www.eBizMBA.com; 2015

\bibitem {Google}Sullivan, D. \textit{Google Still World’s Most Popular Search Engine By Far, But Share Of Unique Searchers Dips Slightly},  http://searchengineland.com/google-worlds-most-popular-search-engine-148089, Retrieved 20 March 2015.  

\bibitem {Vaishnavi1}Sandeep Vaishnavi1 , Jesse Calhou , and Anjan Chatterjee (2001). Binding Personal and Peripersonal Space: Evidence from Tactile Extinction. \textit{Journal of Cognitive Neuroscience 13}, 2, pp. 181–189

\bibitem {adam}Lella, A., (2014). comScore Releases March 2014 U.S. Search Engine Rankings. ComScore.com. Retrieved 21 Feb 2015

\bibitem{bowler}Dermot M. Bowler, Sebastian B. Gaigg, John M. Gardiner (2014) Binding of Multiple Features in Memory by High-Functioning Adults with Autism Spectrum Disorder, \textit{Journal of Autism and Developmental Disorders September 2014, 44}, Issue 9, pp 2355-2362

\bibitem {googlebing} Parrack, D.\textit{4 Search Engines That Combine Google \& Bing}, http://www.makeuseof.com/tag/4-search-engines-that-combine-google-bing/, Retrieved 6 April 2015.


\bibitem{faroo}Faroo, \textit{Free Faroo API}, http://www.faroo.com/hp/api/api.html, Retrieved 6 April 2015.

\bibitem{mottron} Laurent Mottron, Jacob A. Burack, Johannes E. A. Stauder, Philippe Robaey (1999) Perceptual Processing among High-functioning Persons with Autism. \textit{Journal of Child Psychology and Psychiatry 40}, (2), 203–211. doi:10.1111/1469-7610.00433

\end{thebibliography}
\end{document}
